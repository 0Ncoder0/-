\documentclass{article}
\usepackage{circuitikz}

\begin{document}
\begin{circuitikz}
    \draw
        % 输入标记
        (0,4) node[left] {A}
        (0,3) node[left] {B}
        (0,1) node[left] {$C_{in}$}
        
        % 异或门
        (2,3.5) node[xor port] (xor1) {XOR1}
        (4,2.5) node[xor port] (xor2) {XOR2}
        
        % 与门
        (2,1.5) node[and port] (and1) {AND1}
        (4,0.5) node[and port] (and2) {AND2}
        
        % 或门
        (6,1) node[or port] (or1) {OR1}
        
        % 连线
        (0,4) -- (xor1.in 1)
        (0,3) -- (xor1.in 2)
        (xor1.out) -- (xor2.in 1)
        (0,1) -- (xor2.in 2)
        (xor2.out) -- (7,2.5) node[right] {S}
        
        % 进位输出连线
        (or1.out) -- (7,1) node[right] {$C_{out}$}
        
        % 补充缺失的连线
        (0,4) -| (and1.in 1)
        (0,3) -| (and1.in 2)
        (xor1.out) -| (and2.in 1)
        (0,1) -| (and2.in 2)
        (and1.out) -| (or1.in 1)
        (and2.out) -| (or1.in 2);
\end{circuitikz};
\end{document}